%
% circlething.tex
% Copyright 2011, Stefan Beller <stefanbeller@googlemail.com>
%
% This file is part of texbox.
%
% This program is free software; you can redistribute it and/or modify it
% under the terms of the GNU General Public License as published by the Free
% Software Foundation; either version 2 of the License, or (at your option)
% any later version.
%
% This program is distributed in the hope that it will be useful, but WITHOUT
% ANY WARRANTY; without even the implied warranty of MERCHANTABILITY or
% FITNESS FOR A PARTICULAR PURPOSE. See the GNU General Public License for
% more details.
%
% You should have received a copy of the GNU General Public License along with
% this program. If not, see <http://www.gnu.org/licenses/>.
%


% args:
% #1 position as tuple x,y
% #2 name of object
% #3 input directions: N, S, W, E, if small letters add minus sign
% #4 output directions: N, S, W, E
% #5 text
\newcommand{\CIRCLETHING}[5]
{
    \path (#1) coordinate (COORDINATE);
    \draw (COORDINATE)  circle (.1);

    \path (COORDINATE) ++( 0, 0.3) coordinate (#2_INN);
    \path (COORDINATE) ++( 0,-0.3) coordinate (#2_INS);
    \path (COORDINATE) ++(-0.3, 0) coordinate (#2_INW);
    \path (COORDINATE) ++( 0.3, 0) coordinate (#2_INE);

    \path (#2_INN) coordinate (#2_OUTN);
    \path (#2_INS) coordinate (#2_OUTS);
    \path (#2_INW) coordinate (#2_OUTW);
    \path (#2_INE) coordinate (#2_OUTE);

    \path (COORDINATE) ++( 0,  0.1) coordinate (lineN);
    \path (COORDINATE) ++( 0, -0.1) coordinate (lineS);
    \path (COORDINATE) ++( -0.1, 0) coordinate (lineW);
    \path (COORDINATE) ++(  0.1, 0) coordinate (lineE);

    \path (COORDINATE) ++( -0.15, 0.2) coordinate (INNW);
    \path (COORDINATE) ++(  0.2, 0.15) coordinate (INNE);
    \path (COORDINATE) ++( -0.2,-0.15) coordinate (INSW);
    \path (COORDINATE) ++(  0.15,-0.2) coordinate (INSE);

    \draw (COORDINATE) node[draw=none,fill=none]{ #5 };

    \findchars[q]{#3}{N};
    \ifnum \theresult>0
        \draw [->, thick] (#2_INN) -- (lineN);
    \fi
    \findchars[q]{#3}{S};
    \ifnum \theresult>0
        \draw [->, thick] (#2_INS) -- (lineS);
    \fi
    \findchars[q]{#3}{W};
    \ifnum \theresult>0
        \draw [->, thick] (#2_INW) -- (lineW);
    \fi
    \findchars[q]{#3}{E};
    \ifnum \theresult>0
        \draw [->, thick] (#2_INE) -- (lineE);
    \fi
    % subtraktion durch kleinbuchstaben
    \findchars[q]{#3}{n};
    \ifnum \theresult>0
        \draw [->, thick] (#2_INN) -- (lineN);
        \draw (INNW) node[draw=none,fill=none]{ - };
    \fi
    \findchars[q]{#3}{s};
    \ifnum \theresult>0
        \draw [->, thick] (#2_INS) -- (lineS);
        \draw (INSE) node[draw=none,fill=none]{ - };
    \fi
    \findchars[q]{#3}{w};
    \ifnum \theresult>0
        \draw [->, thick] (#2_INW) -- (lineW);
        \draw (INSW) node[draw=none,fill=none]{ - };
    \fi
    \findchars[q]{#3}{e};
    \ifnum \theresult>0
        \draw [->, thick] (#2_INE) -- (lineE);
        \draw (INNE) node[draw=none,fill=none]{ - };
    \fi

    \findchars[q]{#4}{N};
    \ifnum \theresult>0
        \draw [-, thick] (#2_INN) -- (lineN);
    \fi
    \findchars[q]{#4}{S};
    \ifnum \theresult>0
        \draw [-, thick] (#2_INS) -- (lineS);
    \fi
    \findchars[q]{#4}{W};
    \ifnum \theresult>0
        \draw [-, thick] (#2_INW) -- (lineW);
    \fi
    \findchars[q]{#4}{E};
    \ifnum \theresult>0
        \draw [-, thick] (#2_INE) -- (lineE);
    \fi
}

% 3 std args, ins and outs with N S W E !
\newcommand{\ADD}[4]
{
    \CIRCLETHING{#1}{#2}{#3}{#4}{+}
}

% 3 std args, ins and outs with N S W E !
\newcommand{\MUL}[4]
{
    \CIRCLETHING{#1}{#2}{#3}{#4}{$\cdot$}
}
