%
% connections.tex
% Copyright 2011, Jessica Tölke <jtoelke@gmx.com>
%
% This file is part of texbox.
%
% This program is free software; you can redistribute it and/or modify it
% under the terms of the GNU General Public License as published by the Free
% Software Foundation; either version 2 of the License, or (at your option)
% any later version.
%
% This program is distributed in the hope that it will be useful, but WITHOUT
% ANY WARRANTY; without even the implied warranty of MERCHANTABILITY or
% FITNESS FOR A PARTICULAR PURPOSE. See the GNU General Public License for
% more details.
%
% You should have received a copy of the GNU General Public License along with
% this program. If not, see <http://www.gnu.org/licenses/>.
%

% arguments
% #1 coordinates
% #2 name
% #3 label
% resistor in vertical direction with label to the right
\newcommand{\RESISTORV}[3]
{
    \BOX{#1}{#2}{}{0.2,0.6}{}{NS};
    \draw (#2_OUTE) node[draw=none,fill=none]{ \tiny{#3} };
}

% arguments
% #1 coordinates
% #2 name
% #3 label
% resistor in horizontal direction with label above
\newcommand{\RESISTORH}[3]
{
    \BOX{#1}{#2}{}{0.6,0.2}{}{WE};
    \draw (#2_OUTN) node[draw=none,fill=none]{ \tiny{#3} };
}

% arguments
% #1 coordinates
% #2 name
% #3 label
% capacitor in vertical direction with label to the right
\newcommand{\CAPACITORV}[3]
{
    \path (#1) ++ (0,0.4) coordinate(#2_OUTN);
    \path (#1) ++ (0,-0.4) coordinate(#2_OUTS);
    \path (#1) ++ (-0.4,0) coordinate(#2_OUTW);
    \path (#1) ++ (0.4,0) coordinate(#2_OUTE);

    \path (#1) ++ (-0.2,0.05) coordinate(upperleft);
    \path (#1) ++ (0.2,0.05) coordinate(upperright);
    \path (#1) ++ (-0.2,-0.05) coordinate(lowerleft);
    \path (#1) ++ (0.2,-0.05) coordinate(lowerright);
    \path (#1) ++ (0,0.05) coordinate(uppermiddle);
    \path (#1) ++ (0,-0.05) coordinate(lowermiddle);

    \draw (uppermiddle) -- (#2_OUTN);
    \draw (lowermiddle) -- (#2_OUTS);
    \draw (upperleft) -- (upperright);
    \draw (lowerleft) -- (lowerright);

    \draw (#2_OUTE) node[draw=none,fill=none]{ \tiny{#3} };
}

% arguments
% #1 coordinates
% #2 name
% #3 label
% capacitor in horizontal direction with label above
\newcommand{\CAPACITORH}[3]
{
    \path (#1) ++ (0,0.4) coordinate(#2_OUTN);
    \path (#1) ++ (0,-0.4) coordinate(#2_OUTS);
    \path (#1) ++ (-0.4,0) coordinate(#2_OUTW);
    \path (#1) ++ (0.4,0) coordinate(#2_OUTE);

    \path (#1) ++ (-0.05,0.2) coordinate(upperleft);
    \path (#1) ++ (0.05,0.2) coordinate(upperright);
    \path (#1) ++ (-0.05,-0.2) coordinate(lowerleft);
    \path (#1) ++ (0.05,-0.2) coordinate(lowerright);
    \path (#1) ++ (-0.05,0) coordinate(leftmiddle);
    \path (#1) ++ (0.05,0) coordinate(rightmiddle);

    \draw (#2_OUTW) -- (leftmiddle);
    \draw (#2_OUTE) -- (rightmiddle);
    \draw (upperleft) -- (lowerleft);
    \draw (upperright) -- (lowerright);

    \draw (#2_OUTN) node[draw=none,fill=none]{ \tiny{#3} };
}
