%
% box.tex
% Copyright 2011, Stefan Beller <stefanbeller@googlemail.com>
%
% This file is part of texbox.
%
% This program is free software; you can redistribute it and/or modify it
% under the terms of the GNU General Public License as published by the Free
% Software Foundation; either version 2 of the License, or (at your option)
% any later version.
%
% This program is distributed in the hope that it will be useful, but WITHOUT
% ANY WARRANTY; without even the implied warranty of MERCHANTABILITY or
% FITNESS FOR A PARTICULAR PURPOSE. See the GNU General Public License for
% more details.
%
% You should have received a copy of the GNU General Public License along with
% this program. If not, see <http://www.gnu.org/licenses/>.
%


% args:
% #1 position as tuple x,y
% #2 name of object
% #3 text
% #4 size tuple x,y which is overall width/height
% #5 input directions: N, S, W, E
% #6 output directions: N, S, W, E
%
\newcommand{\BOX}[6]
{
    \path (#1) coordinate (COORDINATE);
    \draw (COORDINATE) node[draw=none,fill=none]{ #3 };

    \path let \p1 = (#4) in
        \pgfextra{\pgfmathparse{0.5*(\x1)}}
        (\pgfmathresult pt ,0) coordinate (halfwidth);
    \path let \p1 = (#4) in
        \pgfextra{\pgfmathparse{0.5*(\x1)}}
        (-\pgfmathresult pt ,0) coordinate (nhalfwidth);
    \path let \p1 = (#4) in
        \pgfextra{\pgfmathparse{0.5*(\y1)}}
        (0,\pgfmathresult pt ) coordinate (halfheight);
    \path let \p1 = (#4) in
        \pgfextra{\pgfmathparse{0.5*(\y1)}}
        (0,-\pgfmathresult pt ) coordinate (nhalfheight);

    \draw (COORDINATE) ++(halfwidth) ++(halfheight) coordinate (cornera);
    \draw (COORDINATE) ++(nhalfwidth) ++(nhalfheight) coordinate (cornerb);
    \draw (cornera) rectangle (cornerb);

    \path (COORDINATE) ++(halfheight) coordinate (lineN);
    \path (COORDINATE) ++(nhalfheight) coordinate (lineS);
    \path (COORDINATE) ++(nhalfwidth) coordinate (lineW);
    \path (COORDINATE) ++(halfwidth) coordinate (lineE);

    \path (lineN) ++( 0, 0.3) coordinate (#2_INN);
    \path (lineS) ++( 0,-0.3) coordinate (#2_INS);
    \path (lineW) ++(-0.3, 0) coordinate (#2_INW);
    \path (lineE) ++( 0.3, 0) coordinate (#2_INE);

    \path (#2_INN) coordinate (#2_OUTN);
    \path (#2_INS) coordinate (#2_OUTS);
    \path (#2_INW) coordinate (#2_OUTW);
    \path (#2_INE) coordinate (#2_OUTE);

    \findchars[q]{#5}{N};
    \ifnum \theresult>0
        \draw [->, thick] (#2_INN) -- (lineN);
    \fi

    \findchars[q]{#5}{S};
    \ifnum \theresult>0
        \draw [->, thick] (#2_INS) -- (lineS);
    \fi

    \findchars[q]{#5}{W};
    \ifnum \theresult>0
        \draw [->, thick] (#2_INW) -- (lineW);
    \fi

    \findchars[q]{#5}{E};
    \ifnum \theresult>0
        \draw [->, thick] (#2_INE) -- (lineE);
    \fi

    \findchars[q]{#6}{N};
    \ifnum \theresult>0
        \draw [-, thick] (#2_INN) -- (lineN);
    \fi
    \findchars[q]{#6}{S};
    \ifnum \theresult>0
        \draw [-, thick] (#2_INS) -- (lineS);
    \fi
    \findchars[q]{#6}{W};
    \ifnum \theresult>0
        \draw [-, thick] (#2_INW) -- (lineW);
    \fi
    \findchars[q]{#6}{E};
    \ifnum \theresult>0
        \draw [-, thick] (#2_INE) -- (lineE);
    \fi
}

\newcommand{\BOXUP}[3]
{
\BOX{#1}
}
