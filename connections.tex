%
% connections.tex
% Copyright 2011, Stefan Beller <stefanbeller@googlemail.com>
% Copyright 2011, Jessica Tölke <jtoelke@gmx.com>
%
% This file is part of texbox.
%
% This program is free software; you can redistribute it and/or modify it
% under the terms of the GNU General Public License as published by the Free
% Software Foundation; either version 2 of the License, or (at your option)
% any later version.
%
% This program is distributed in the hope that it will be useful, but WITHOUT
% ANY WARRANTY; without even the implied warranty of MERCHANTABILITY or
% FITNESS FOR A PARTICULAR PURPOSE. See the GNU General Public License for
% more details.
%
% You should have received a copy of the GNU General Public License along with
% this program. If not, see <http://www.gnu.org/licenses/>.
%


% arguments
% #1 coordinate
% #2 coordinate
% draws a direct connection between given coordinates
\newcommand{\CONNECT}[2]
{
    \draw [-] (#1) -- (#2);
}

% arguments
% #1 coordinate
% #2 coordinate
% draws a rectangular connection between given coordinates,
% first in vertical direction, then horizontal
\newcommand{\CONNECTVH}[2]
{
    \draw
      let
        \p1=(#1),
        \p2=(#2)
      in
        [-] (\x1, \y1) -- (\x1, \y2) -- (\x2, \y2);
}

% arguments
% #1 coordinate
% #2 coordinate
% draws a rectangular connection between given coordinates,
% first in vertical direction, then horizontal
% ending with an arrow
\newcommand{\ACONNECTVH}[2]
{
    \draw
      let
        \p1=(#1),
        \p2=(#2)
      in
        [->] (\x1, \y1) -- (\x1, \y2) -- (\x2, \y2);
}

% arguments
% #1 coordinate
% #2 coordinate
% draws a rectangular connection between given coordinates,
% first in horizontal direction, then vertical
\newcommand{\CONNECTHV}[2]
{
    \draw
      let
        \p1=(#1),
        \p2=(#2)
      in
        [-] (\x1, \y1) -- (\x2, \y1) -- (\x2, \y2);
}

% arguments
% #1 coordinate
% #2 coordinate
% draws a rectangular connection between given coordinates,
% first in horizontal direction, then vertical
% ending with an arrow
\newcommand{\ACONNECTHV}[2]
{
    \draw
      let
        \p1=(#1),
        \p2=(#2)
      in
        [->] (\x1, \y1) -- (\x2, \y1) -- (\x2, \y2);
}

% arguments
% #1 coordinate
% #2 name
% a filled dot that can be connected
\newcommand{\DOT}[2]
{
    \filldraw (#1) circle (0.03);
    \path (#1) coordinate(#2);
}

% arguments
% #1 coordinate
% #2 name
% an empty dot that can be connected at N, S, W, E
\newcommand{\OPENCONNECTOR}[2]
{
    \draw (#1) circle (0.03);
    \path(#1) ++ (0,0.03) coordinate(#2_N);
    \path(#1) ++ (0,-0.03) coordinate(#2_S);
    \path(#1) ++ (-0.03,0) coordinate(#2_W);
    \path(#1) ++ (0.03,0) coordinate(#2_E);
}
